\documentclass{article}
\usepackage{geometry}
\usepackage{fancyhdr}
\usepackage{ctex}
\usepackage{lastpage}
\usepackage{tikz}
\usepackage{ifthen}

% 页面设置
\geometry{a4paper, margin=1in}

% 密封线命令
\newcommand{\sealline}[1]{
  \begin{tikzpicture}[remember picture,overlay]
    \ifthenelse{\isodd{\thepage}}{
      % 奇数页密封线在左侧
      \draw[thick] (current page.north west) ++(0,-2) -- (current page.south west) ++(0,2);
      \node[rotate=90,anchor=center] at (current page.west) {#1};
    }{
      % 偶数页密封线在右侧
      \draw[thick] (current page.north east) ++(0,-2) -- (current page.south east) ++(0,2);
      \node[rotate=-90,anchor=center] at (current page.east) {#1};
    }
  \end{tikzpicture}
}

% 页眉页脚设置
\pagestyle{fancy}
\fancyhf{}
\fancyhead[C]{\{HEADER_CONTENT\}}
\fancyfoot[C]{第\thepage 页/共\pageref{LastPage}页}

\begin{document}

% 抬头部分
\begin{center}
{\Large \textbf{\{EXAM_TITLE\}}}\\
\vspace{0.5cm}
考试科目:\{SUBJECT\}\\
考试时间:\{EXAM_TIME\}分钟\\
总分:\{TOTAL_POINTS\}分
\end{center}

% 密封线
\sealline{
  \begin{tabular}{ll}
    姓名:\underline{\hspace{3cm}} & 考号:\underline{\hspace{3cm}} \\
    班级:\underline{\hspace{3cm}} & 日期:\underline{\hspace{3cm}} \\
  \end{tabular}
}

\vspace{1cm}

% 题目内容
\{CONTENT\}

\end{document}